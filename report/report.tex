\documentclass[14pt,a4paper]{extarticle}
\usepackage[utf8]{inputenc}
\usepackage[russian]{babel}
\usepackage{fontspec}
\setmainfont{Times New Roman}
\usepackage{geometry}
\geometry{left=3cm,right=2cm,top=2cm,bottom=2cm}
\usepackage{setspace}
\setstretch{1.0}
\usepackage{indentfirst}
\setlength{\parindent}{1.25cm}
\usepackage{graphicx}
\usepackage{hyperref}
\hypersetup{
    colorlinks=true,
    linkcolor=black,
    urlcolor=blue,
    citecolor=black
}

\begin{document}

% Титульный лист
\begin{titlepage}
\centering
\large
\textbf{ВЫСШАЯ ШКОЛА ЭКОНОМИКИ}\\[0.5cm]
\textbf{Факультет компьютерных наук}\\[0.5cm]
Магистерская программа «Исследования и предпринимательство в искусственном интеллекте»\\[3cm]

\Large
\textbf{КОРПУСНЫЙ АНАЛИЗ АРХАИЗАЦИИ РУССКОЙ ПОЭТИЧЕСКОЙ ЛЕКСИКИ}\\[1cm]

\large
Исследовательская работа\\[3cm]

\raggedleft
\begin{tabular}{l}
Выполнил:\\
студент 2 курса\\
Пеганов Никита\\[1cm]
\end{tabular}

\vfill

\centering
Москва 2025
\end{titlepage}

% Введение
\section*{Введение}
\addcontentsline{toc}{section}{Введение}

\textbf{Актуальность.} Изучение динамики языковых изменений в художественной литературе является важной задачей исторической лексикологии и корпусной лингвистики. Архаизмы – устаревшие слова и выражения, вытесненные из активного употребления синонимами, – представляют особый интерес как маркеры стилистических и эстетических установок литературных эпох. Процесс архаизации поэтической лексики отражает не только естественную эволюцию языка, но и сознательный выбор авторов, их отношение к традиции и поиск выразительных средств.

Количественный анализ частотности архаизмов в диахронии позволяет выявить закономерности их использования, связанные со сменой литературных направлений, и проверить гипотезы о корреляции между эстетическими парадигмами и лексическим составом текстов.

\textbf{Цель работы} – проанализировать динамику использования архаичной и устаревшей лексики в русской поэзии XVIII–XX веков на материале корпуса из 16 694 стихотворений.

\textbf{Задачи исследования:}
\begin{enumerate}
\item Подготовить корпус стихотворений с метаданными (автор, год написания).
\item Составить словарь архаизмов на основе лексикографических источников.
\item Вычислить частотность архаизмов по десятилетиям и литературным течениям.
\item Выявить периоды интенсивной архаизации и интерпретировать полученные результаты.
\item Установить корреляцию между использованием архаизмов и литературными направлениями.
\end{enumerate}

\textbf{Материал исследования} – корпус русской поэзии, включающий 16 694 стихотворения 195 авторов, созданных в период с 1725 по 1996 год. Общий объем корпуса составил 1 726 105 словоупотреблений. Корпус был получен из открытого проекта «Поэтический корпус русского языка» (Гусев 2016), доступного на платформе GitHub. Исходные данные представлены в формате XML и включают метаданные: имя автора, название стихотворения, даты создания, тематические теги.

\textbf{Методы исследования.} В работе применялись методы корпусной лингвистики и квантитативного анализа текстов. На первом этапе был выполнен парсинг XML-файла и конвертация данных в табличный формат (CSV, Parquet). Токенизация текстов проводилась с помощью регулярных выражений для извлечения русских слов. 

Словарь архаизмов был составлен на основе «Словаря устаревших слов» с портала «Азбука веры» (1 188 слов), дополнен 18 лексемами, выявленными при визуальном анализе облаков слов, построенных по частотности для XVIII, XIX и XX веков. Для вычисления относительной частотности использовалась метрика «количество архаизмов на 1000 словоупотреблений». 

Периодизация материала осуществлялась по десятилетиям и литературным течениям (классицизм, сентиментализм, романтизм, реализм, символизм, акмеизм, футуризм, советская поэзия). Визуализация результатов выполнена с использованием библиотек Python (matplotlib, seaborn, wordcloud).

\section*{Основная часть}
\addcontentsline{toc}{section}{Основная часть}

\subsection*{Подготовка данных}

Исходный XML-файл был распарсен с извлечением полей: автор, название, текст стихотворения, годы написания (date\_from, date\_to), темы. Для каждого стихотворения был вычислен средний год создания. После фильтрации по наличию даты в анализ вошло 12 857 текстов (77\% корпуса), охватывающих период 1725–1996 годов.

Токенизация текстов производилась с выделением последовательностей русских букв длиной не менее 3 символов. Словарь архаизмов был загружен в виде множества лемм для быстрого поиска. Дополнительно был реализован скрипт для визуального анализа наиболее частотных и редких слов по векам, что позволило выявить архаизмы, отсутствовавшие в исходном словаре (например: \textit{ныне, взор, глас, коль, старец}).

\subsection*{Результаты количественного анализа}

\textbf{Общая статистика.} В корпусе обнаружено 14 922 употребления архаизмов, что составляет среднюю частотность 8,64 архаизма на 1000 словоупотреблений. Максимальная частотность зафиксирована в 1720-х годах – 27,69 на 1000 слов, минимальная – в 1990-х годах (5,92 на 1000). Общее снижение частотности за 270 лет составило 78,6\%.

\textbf{Динамика по десятилетиям.} Анализ показал устойчивый тренд снижения частотности архаизмов с XVIII по XX век. Наибольшие значения характерны для периода 1720–1770 годов (от 27,69 до 14,86 на 1000 слов). К концу XIX века частотность снижается до 8–9 на 1000, в XX веке – до 6–7 на 1000 слов.

Важно отметить \textbf{циклический паттерн}: наблюдается локальный рост частотности архаизмов к концу каждой литературной эпохи. Пики зафиксированы в конце классицизма (~1800-е), сентиментализма (~1810-е), романтизма (~1830-е) и реализма (~1890-е). Этот паттерн отсутствует у модернистских течений (символизм, футуризм, советская поэзия), где наблюдается стабильное снижение без циклических колебаний.

\textbf{Анализ по литературным течениям.} Для каждого течения была вычислена средняя частотность архаизмов (таблица 1).

\begin{table}[h]
\centering
\caption{Частотность архаизмов по литературным течениям}
\begin{tabular}{|l|c|c|}
\hline
\textbf{Литературное течение} & \textbf{Период} & \textbf{Частотность (на 1000 слов)} \\
\hline
Классицизм & 1730–1800 & 16,93 \\
Сентиментализм & 1770–1820 & 13,17 \\
Романтизм & 1800–1840 & 11,68 \\
Реализм & 1840–1890 & 8,70 \\
Символизм & 1890–1910 & 8,38 \\
Акмеизм & 1910–1920 & 7,70 \\
Футуризм & 1910–1930 & 7,53 \\
Советская поэзия & 1920–1990 & 6,62 \\
\hline
\end{tabular}
\end{table}

Стандартное отклонение между течениями составило $\sigma$ = 3,55, что свидетельствует о статистически значимом влиянии литературного направления на частотность архаизмов. Максимальная разница между классицизмом и советской поэзией достигает 10,31 единицы (разница в 2,6 раза).

\textbf{Анализ по авторам.} Среди топ-10 авторов по количеству стихотворений наибольшую частотность архаизмов демонстрируют представители романтизма: Михаил Лермонтов (12,81 на 1000 слов), Федор Тютчев (12,44), Александр Пушкин (11,42). Минимальные значения – у поэтов XX века: Анна Ахматова (5,48) и Владимир Высоцкий (5,51).

\subsection*{Визуализация результатов}

На рисунке 1 представлена динамика частотности архаизмов по десятилетиям с 1720 по 2000 год. График демонстрирует общий нисходящий тренд с локальными пиками в конце каждой литературной эпохи до конца XIX века.

\begin{figure}[h]
\centering
\includegraphics[width=0.85\textwidth]{../results/archaism_dynamics_by_decade.png}
\caption{Динамика частотности архаизмов по десятилетиям (1720–2000)}
\end{figure}

~\\~\\~\\~\\~\\~\\

Рисунок 2 показывает сравнение средней частотности архаизмов по литературным течениям. Чётко видна градация от классицизма (максимум) к советской поэзии (минимум).

\begin{figure}[h]
\centering
\includegraphics[width=0.8\textwidth]{../results/literary_movements_comparison.png}
\caption{Сравнение литературных течений по частотности архаизмов}
\end{figure}
~\\~\\~\\~\\~\\

На рисунке 3 представлена временная шкала, совмещающая динамику частотности с периодами литературных течений. Цветные полосы иллюстрируют хронологические рамки каждого направления.

\begin{figure}[h]
\centering
\includegraphics[width=0.8\textwidth]{../results/timeline_with_movements.png}
\caption{Временная шкала частотности архаизмов с литературными течениями}
\end{figure}
~\\~\\

Рисунок 4 демонстрирует сравнение частотности архаизмов по векам (XVIII, XIX, XX). Заметно прогрессивное снижение показателя от 18,4 (XVIII век) до 7,0 (XX век).

\begin{figure}[h]
\centering
\includegraphics[width=0.75\textwidth]{../results/archaism_by_century.png}
\caption{Сравнение частотности архаизмов по векам}
\end{figure}

\clearpage

\subsection*{Интерпретация результатов}

Полученные данные подтверждают гипотезу о снижении частотности архаизмов в русской поэзии с XVIII по XX век. Этот процесс обусловлен несколькими факторами.

Во-первых, \textbf{естественное устаревание лексики}. Часть слов, использовавшихся поэтами XVIII–XIX веков как нейтральная или современная лексика, к концу XX века перешла в разряд архаизмов в результате естественной эволюции языка. Таким образом, рост доли архаизмов в более ранних текстах отчасти является артефактом исторической дистанции: чем старше текст, тем больше вероятность, что его лексика устарела к моменту анализа (конец XX – начало XXI века).

Во-вторых, эволюция литературного языка в сторону сближения с разговорной нормой и отказ от церковнославянской книжной традиции. Классицизм, ориентированный на античные образцы и высокий стиль, активно использовал старославянизмы как маркеры торжественности. Романтизм сохранил интерес к архаике через обращение к исторической и фольклорной тематике.

В-третьих, смена эстетических парадигм в конце XIX – начале XX века. Модернистские течения (символизм, футуризм) провозгласили разрыв с традицией, что привело к минимизации использования устаревшей лексики. Советская поэзия продолжила этот тренд, тяготея к современному урбанизированному языку.

В-четвертых, циклический паттерн роста архаизмов к концу каждой эпохи может быть интерпретирован как «прощание с традицией» перед сменой литературной парадигмы. Поэты переходного периода обращались к архаичной лексике, чтобы подчеркнуть связь с уходящей эпохой и её эстетическими идеалами. Отсутствие этого паттерна у модернистов указывает на более радикальный характер разрыва с прошлым.

Корреляция между частотностью архаизмов и литературным течением ($\sigma$ = 3,55) свидетельствует о том, что архаизация была не пассивным следом традиции, а сознательным \textbf{эстетическим выбором}, соответствующим идеологии и стилистике направления.

\section*{Заключение}
\addcontentsline{toc}{section}{Заключение}

В ходе исследования был проведен корпусный анализ частотности архаизмов в русской поэзии XVIII–XX веков на материале 16 694 стихотворений (1 726 105 словоупотреблений). Основные результаты работы:

\begin{enumerate}
\item Установлено устойчивое снижение частотности архаизмов на 78,6\% за период 1720–1990 годов (с 27,69 до 5,92 на 1000 слов).

\item Выявлена статистически значимая корреляция между литературным течением и уровнем архаизации ($\sigma$ = 3,55). Классицизм демонстрирует максимальную частотность (16,93 на 1000), советская поэзия – минимальную (6,62 на 1000).

\item Обнаружен циклический паттерн: локальный рост частотности архаизмов к концу каждой литературной эпохи (классицизм, сентиментализм, романтизм, реализм). Этот паттерн отсутствует у модернистских течений, что указывает на принципиально иной характер отношения к языковой традиции.

\item Среди авторов наибольшую частотность архаизмов демонстрируют представители романтизма: М. Лермонтов (12,81), Ф. Тютчев (12,44), А. Пушкин (11,42).
\end{enumerate}

Полученные результаты позволяют утверждать, что использование архаизмов в поэзии было не механическим наследованием традиции, а осознанным стилистическим приёмом, коррелирующим с эстетическими установками литературных направлений. Циклический паттерн архаизации свидетельствует о рефлексивном отношении поэтов к смене эпох.

Перспективы исследования включают расширение словаря архаизмов с учётом морфологических форм, анализ контекстов употребления устаревшей лексики и сопоставление с прозаическими текстами.

\section*{Список литературы}
\addcontentsline{toc}{section}{Список литературы}

\begin{enumerate}
\item Виноградов В. В. Очерки по истории русского литературного языка XVII–XIX веков. М.: Высшая школа, 1982.

\item Грановская Л. М. Русский литературный язык в конце XIX и XX вв. М.: Элпис, 2005.

\item Гусев И. О. Поэтический корпус русского языка // GitHub repository. 2016. URL: https://github.com/IlyaGusev/PoetryCorpus (дата обращения: 20.12.2025).

\item Словарь устаревших слов // Азбука веры. URL: https://azbyka.ru/otechnik/\\
Spravochniki/slovar-ustarevshih-slov/ (дата обращения: 20.12.2025).

\item Пеганов Н. С. Корпусный анализ архаизации русской поэтической лексики: исходный код исследования // GitHub repository. 2025. URL: https://github.com/NikPeg/poetry-archaization-analysis (дата обращения: 24.12.2025).

\item Успенский Б. А. Краткий очерк истории русского литературного языка (XI–XIX вв.). М.: Гнозис, 1994.

\item Lyashevskaya O., Sharov S. Frequency dictionary of modern Russian: the Russian National Corpus // Online: http://dict.ruslang.ru/freq.php, 2009.
\end{enumerate}

\end{document}

